%!TEX root = ../main.tex
\titleformat{\chapter}[hang]
    {\normalfont\Huge\bfseries}{\chaptertitlename\ \thechapter\ -\ }{0pt}{\Huge}
\titlespacing*{\chapter}{0pt}{0pt}{20pt}



\chapter*{Abstract}
\thispagestyle{fancy} %DO NOT REMOVE THIS LINE
\addcontentsline{toc}{chapter}{Abstract}

[The text within the square brackets must be deleted along with the square brackets when finalising your own abstract.

The abstract for an undergraduate dissertation should be between 200 - 350 words.

Arial, Normal, 11pt with 1.2 or 1.5 line spacing should be used. The text in this part has 1.5 line spacing.

An abstract is a brief, accurate and comprehensive summary of the entire dissertation. It is the first thing to be read by your examiners to help them know the brief content of the dissertation. It also serves as a “sales pitch” to form the first impression of your work.

A good abstract should be accurate, self-contained, concise, specific and clear. A quick way to assess the quality of your abstract is to check whether it answers the questions why, how, what and so what.

Researching the efficacy of using CNN's (Convolutional neural networks to identify crop defects) and creating a suitable platform for users to interact with the network.

\textbf{It is easier to write the Abstract the last.}]





\chapter*{Dissertation Declaration}
\thispagestyle{fancy} %DO NOT REMOVE THIS LINE

\color{red}
[The text within the square brackets must be deleted along with the square brackets when finalising your declaration.

Note if your project is CONFIDENTIAL because of your client, you will need to adapt this declaration based on the agreement between you and your client accordingly. Do not forget to state the name of your client clearly. You must contact and inform Project Coordinator if your project is CONFIDENTIAL.]
\color{black}

{\linespread{1.0} %The BU template dictates this to be this line spacing
I agree that, should the University wish to retain it for reference purposes, a copy of my dissertation may be held by Bournemouth University normally for a period of 3 academic years. I understand that once the retention period has expired my dissertation will be destroyed.

\section*{Confidentiality}
I confirm that this dissertation does not contain information of a commercial or confidential nature or include personal information other than that which would normally be in the public domain unless the relevant permissions have been obtained. In particular any information which identifies a particular individual's religious or political beliefs, information relating to their health, ethnicity, criminal history or sex life has been anonymised unless permission has been granted for its publication from the person to whom it relates.

\section*{Copyright}
The copyright for this dissertation remains with me.

\section*{Requests for Information}
I agree that this dissertation may be made available as the result of a request for information under the Freedom of Information Act.


\vspace{12pt}
\textbf{Signed:}
\hrule width 0.7\textwidth
\vspace{12pt}

%TODO: FILL IN ACCORDINGLY
Name: [Your name]
\vspace{3pt}

Date: [Date of signing this declaration]
\vspace{3pt}

Programme: [Your degree title]
\vspace{3pt}

}







\chapter*{Original Work Declaration}
\thispagestyle{fancy} %DO NOT REMOVE THIS LINE


This dissertation and the project that it is based on are my own work, except where stated, in accordance with University regulations.

\vspace{12pt}
\textbf{Signed:}
\hrule width 0.7\textwidth
\vspace{12pt}

%TODO: FILL IN ACCORDINGLY
Name: [Your name]
\vspace{3pt}

Date: [Date of signing this declaration]



\chapter*{Acknowledgements}
\thispagestyle{fancy} %DO NOT REMOVE THIS LINE
\addcontentsline{toc}{chapter}{Acknowledgements}

[The text within the square brackets must be deleted along with the square brackets when finalising your own acknowledgements.

Arial, Normal, 11pt with 1.2 or 1.5 line spacing should be used. The text in this part has 1.5 line spacing.

This is your opportunity to mention individuals who have been particularly helpful. Reading the acknowledgements in the past dissertations in the project library will give you an idea of the ways in which different kinds of help have been appreciated and mentioned.]

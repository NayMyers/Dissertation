\chapter{Introduction}
\label{introduction}

\section{Context}
  With the increased availability of smartphones \cite{Statista:2021}, digital cameras \cite{ImarcGroup} and Internet access \cite{Wikipedia} \cite{Globaltt}. Coupled with the increased interest in home food cultivation \cite{Google} and the large number of people reliant on food grown in smallholdings \cite{JLIFADSmallHolders}. The ability to identify defects with crops using technology has potential to be impactful to many people.
\section{Literature Review}
  \par
  In 2009 a study was conducted using deep learning to identify three different disease classes on rice plants. The results showed over 70\% classification accuracy on 50 sample images. \cite{Anthonys2009}
  \par
   (14 crop species and 26 diseases). It was found that GoogLeNet performed best, acheiving 99.35\% on held out test set.
  \par
  seven years later and there have been great successes in identifying crop disease with CNN's. In 2016 a paper was published running experiments on a 38 class crop disease
  dataset over 14 crop species and 26 diseases (or absense thereof). Resulting in 99.35\% accuracy on a held-out test set (using GoogLeNet). \cite{Mohanty2016}. This study utilized two established CNN architectures, namely AlexNet \cite{Krizhevsky} & GoogLeNet. \cite{Szegedy_2015_CVPR} With GoogLeNet achieving a higher F1 score in almost all cases.
  \par
  Then in 2018 InceptionNetV3 is used on a very similar if not the same dataset of 38 class crop diseases (this paper cites the number of crop species to be 13 appose to 14) and 26 diseases. Resulting in a slight increase in accuracy of 0.39\%, so increasing to 99.74\% classification accuracy.
  \par
\section{Problem Definition}
  As it stands there are currently (09/02/2021) no easily found \footnote[1]{(i.e. not present in the first 3 pages of a google search for 'crop defect identification' and 'What's wrong with my crop')} web interfaces for interacting with a crop defect identification service.
  \par
  Although there exists very capable publicly available image classification networks. \cite{Yandex} There is no bespoke application catering soley to crop defect identification, that has the benefit of providing recourse information to the user.

\section{Proposed Solution}
  To provide a web service that interacts with a convolutional neural network (CNN) backend to diagnose crop defects such as, nurturing problems e.g. lack of water/nitrogen/C02, too hot/cold. And external threats such as crop disease/pest infestation. The interface will be simple and intuitive as possible. The UI should minimise points of interaction and streamline the process of uploading a crop image to be analysed.
	The web service will return information regarding the percentage likelihood of each kind of crop defect, including images that are of similar nature to the one analysed.

\section{Aims and Objectives}
  These should be SMART with clear success criteria defined
  specific, measurable, achievable, realistic, Timebound
  \subsection{Aims}
    \begin{itemize}
      \item To aid gardneners and smallholders in identifying crop defects.
      \item To aid gardeners and smallholders in taking relevant recourse.
    \end{itemize}
  \subsection{Objectives}
    \begin{itemize}
      \item Provide a way for a user to upload an image to be analysed.
      \item Display infomration regarding the likelihood of each kind of defect.
      \item Display recourse information alongside defect information.
      \item Have gallery of images filtered by crop and disease type.
    \end{itemize}

\section{Risk Table}
  % \begin{center}
  % \begin{tabular}{c c c c c}
  %   \hline
  %   ID & Name & Likelihood & Impact & Control Mechanisms \\ [0.5ex]
  %   \hline\hline
  %   01 & Improper Time Management & Med/Low & High & Staying On-track to the Gantt Chart \\
  %   02 & HDD/Storage failure & Low & High & All work will be backed up to github. And intermitently to Google Drive \\
  % \end{tabular}
  % \end{center}


  % Table generated by Excel2LaTeX from sheet 'Risks_Table'
\begin{table}[ht]
  \centering
  \caption{Risks Table}
    \begin{tabular}{| c | c | c | c | c |}
    \toprule
    ID & Name & Likelihood & Impact & Control Mechanisms \\
    \midrule
    1     & Improper Time Management & med/low & high  & \multicolumn{1}{l|}{Follow the Gannt chart} \\
    \midrule
    2     & HDD/storage failure & low   & high  & All work will be backed\newline{} up to github \\
    \midrule
    3     & Illness/Injury & med   & med   & Should the need arise I\newline{}will apply for an extention\newline{}to the due date. \\
    \midrule
    4     & RSI (repetetive strain injury) & med   & low   & Work with proper posture\newline{}and set up workstation properly.\newline{}And take frequent breaks \\
    \midrule
    5     & Eye strain & med   & low   & Ensure room is well lit when working\newline{}on a screen.  \\
    \midrule
    6     & Incorrect Task Prioritisation & med   & med   & Iteratively re-asses the work being\newline{}done and compare it to the mark\newline{}scheme. \\
    \midrule
    7     & Postural problems & med   & low   & Work with proper posture\newline{}and set up workstation properly.\newline{}And take frequent breaks \\
    \bottomrule
    \end{tabular}%
  \label{tab:risksTable}%
\end{table}%


  (ID, name, likelihood, impact, control mechanisms / accept)

\section{Overview}
  Introducing rest of dissertation (with cross references to sections)

\chapter{Introduction}
\label{introduction}

\section{Context}
  With the increased availability of smartphones \cite{Statista}, digital cameras \cite{ImarcGroup} and Internet access \cite{Wikipedia} \cite{Globaltt}. Coupled with the increased interest in home food cultivation \cite{Google} and the large number of people reliant on food grown in smallholdings \cite{JLIFADSmallHolders}. The ability to identify defects with crops using technology has potential to be impactful to many people.
\section{Problem Definition}
  As it stands there are currently (09/02/2021) no easily found \footnote[1]{(i.e. not present in the first 3 pages of a google search for 'crop defect identification' and 'What's wrong with my crop')} web interfaces for interacting with a crop defect identification service.
\section{Proposed Solution}
  To provide a web service that interacts with a convolutional neural network (CNN) backend to diagnose crop defects such as, nurturing problems e.g. lack of water/nitrogen/C02, too hot/cold. And external threats such as crop disease/pest infestation. The interface will be simple and intuitive as possible. The UI should minimise points of interaction and streamline the process of uploading a crop image to be analysed.
	The web service will return information regarding the percentage likelihood of each kind of crop defect, including images that are of similar nature to the one analysed.

\section{Aims and Objectives}
  These should be SMART with clear success criteria defined
  \begin{itemize}
  	\item have a working REST API. The API will provide information regarding the likelihood of each kind of crop defect, when served an image via a link to a relational database. In addition to other metrics such as similar images and time to compute. The API will be robust enough to handle the receipt of erroneous requests.
  	\item A python backend that will handle image classification using a CNN.
    \item The CNN should be able to classify at least 7 different defects across at least two different plant species.
    \item The CNN should acheive at least 80\% accuracy at classifying all different classes of defect in a held out test set that contains an equal number of each class.
  	\item A UI that will allow the user to upload an image to be analysed.
  	\item The UI will display information regarding the likelihood of each kind of possible defect.
  	\item To display the relevant images that fit the description of the most likely defects.
  	\item To display recourse information to rectify the defect.
  	\item Collecting, cleaning and pre-processing the image data.
    \item Artificially grow the dataset by performing translations/rotations/adding noise to the images to make the training data more comprehensive.
  	\item Include regularisation techniques to the NN to prevent overfitting.
  \end{itemize}
\section{Risk Table}
  % \begin{center}
  % \begin{tabular}{c c c c c}
  %   \hline
  %   ID & Name & Likelihood & Impact & Control Mechanisms \\ [0.5ex]
  %   \hline\hline
  %   01 & Improper Time Management & Med/Low & High & Staying On-track to the Gantt Chart \\
  %   02 & HDD/Storage failure & Low & High & All work will be backed up to github. And intermitently to Google Drive \\
  % \end{tabular}
  % \end{center}


  % Table generated by Excel2LaTeX from sheet 'Risks_Table'
\begin{table}[ht]
  \centering
  \caption{Risks Table}
    \begin{tabular}{| c | c | c | c | c |}
    \toprule
    ID & Name & Likelihood & Impact & Control Mechanisms \\
    \midrule
    1     & Improper Time Management & med/low & high  & \multicolumn{1}{l|}{Follow the Gannt chart} \\
    \midrule
    2     & HDD/storage failure & low   & high  & All work will be backed\newline{} up to github \\
    \midrule
    3     & Illness/Injury & med   & med   & Should the need arise I\newline{}will apply for an extention\newline{}to the due date. \\
    \midrule
    4     & RSI (repetetive strain injury) & med   & low   & Work with proper posture\newline{}and set up workstation properly.\newline{}And take frequent breaks \\
    \midrule
    5     & Eye strain & med   & low   & Ensure room is well lit when working\newline{}on a screen.  \\
    \midrule
    6     & Incorrect Task Prioritisation & med   & med   & Iteratively re-asses the work being\newline{}done and compare it to the mark\newline{}scheme. \\
    \midrule
    7     & Postural problems & med   & low   & Work with proper posture\newline{}and set up workstation properly.\newline{}And take frequent breaks \\
    \bottomrule
    \end{tabular}%
  \label{tab:risksTable}%
\end{table}%


  (ID, name, likelihood, impact, control mechanisms / accept)

\section{Overview}
  Introducing rest of dissertation (with cross references to sections)

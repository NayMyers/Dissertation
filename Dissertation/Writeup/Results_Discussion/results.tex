\chapter{Results and Discussion}
\label{results_discussion}

\section{Main Results}
  lorem ipsum

\section{Evaluation Results}
  lorem ipsum

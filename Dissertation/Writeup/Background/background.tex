\chapter{Background}
\section{Problem Definition}
  As it stands there are currently (09/02/2021) No results present in the first 3 pages of a google search for 'crop defect identification' and 'What's wrong with my crop' that show web interfaces for interacting with a crop defect identification service.
  \par
  Although there exists very capable, publicly available image classification networks \citep{Yandex}. There is no bespoke application catering soley to crop defect identification, that has the benefit of providing recourse and prevention information to the user.

\section{Proposed Solution}
  To provide a web service that interacts with a convolutional neural network (CNN) back-end\footnote{see Development Methodology} to diagnose crop defects such as, nurturing problems e.g. lack of water/nitrogen/C02, too hot/cold and external threats such as crop disease/pest infestation. The interface will be simple and intuitive as possible. The User Interface (UI) should minimise points of interaction and streamline the process of uploading a crop image to be analysed.
	The web service will return information regarding the percentage likelihood of each kind of crop defect, including images that are of similar nature to the one analysed.

\section{Aims \& Objectives}
  These should be SMART with clear success criteria defined
  specific, measurable, achievable, realistic, Timebound
  \subsection{Aims}
    \begin{itemize}
      \item To aid gardneners and smallholders in identifying crop defects.
      \item To aid gardeners and smallholders in taking relevant recourse.
    \end{itemize}
  \subsection{Objectives}
    \begin{itemize}
      \item Provide a way for a user to upload an image to be analysed.
      \item Display information regarding the likelihood of each kind of defect.
      \item Display recourse information alongside defect information.
      \item Have gallery of images filtered by crop and disease type.
    \end{itemize}


  % \begin{center}
  % \begin{tabular}{c c c c c}
  %   \hline
  %   ID & Name & Likelihood & Impact & Control Mechanisms \\ [0.5ex]
  %   \hline\hline
  %   01 & Improper Time Management & Med/Low & High & Staying On-track to the Gantt Chart \\
  %   02 & HDD/Storage failure & Low & High & All work will be backed up to github. And intermitently to Google Drive \\
  % \end{tabular}
  % \end{center}


  % Table generated by Excel2LaTeX from sheet 'Risks_Table'
\begin{landscape}
  \section{Risk Table}
  \begin{table}[ht]
    \centering
    \caption{Risks Table}
      \begin{tabular}{| c | c | c | c | c |}
      \toprule
      ID & Name & Likelihood & Impact & Control Mechanisms \\
      \midrule
      1     & Improper Time Management & med/low & high  & \multicolumn{1}{l|}{Follow the Gannt chart} \\
      \midrule
      2     & HDD/storage failure & low   & high  & All work will be backed\newline{} up to github \\
      \midrule
      3     & Illness/Injury & med   & med   & Should the need arise I\newline{}will apply for an extention\newline{}to the due date. \\
      \midrule
      4     & RSI (repetetive strain injury) & med   & low   & Work with proper posture\newline{}and set up workstation properly.\newline{}And take frequent breaks \\
      \midrule
      5     & Eye strain & med   & low   & Ensure room is well lit when working\newline{}on a screen.  \\
      \midrule
      6     & Incorrect Task Prioritisation & med   & med   & Iteratively re-asses the work being\newline{}done and compare it to the mark\newline{}scheme. \\
      \midrule
      7     & Postural problems & med   & low   & Work with proper posture\newline{}and set up workstation properly.\newline{}And take frequent breaks \\
      \bottomrule
      \end{tabular}%
    \label{tab:risksTable}%
  \end{table}%
\end{landscape}

\section{Overview}
  Firstly the introduction will provide additional context to the problem domain and show relevant research. Next, the methododology and justification for the choice of methods will explored. Including employed technologies, project management methodoglogies and development methodologies.
